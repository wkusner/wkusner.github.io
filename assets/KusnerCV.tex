%%%%%%%%%%%%%%%%%%%%%%%%%%%%%%%%%%%%%%%%%%%%%%%%%%%%%%%%%%%%%%%%%%%%%%%%
%%%%%%%%%%%%%%%%%%%%%% Simple LaTeX CV Template %%%%%%%%%%%%%%%%%%%%%%%%
%%%%%%%%%%%%%%%%%%%%%%%%%%%%%%%%%%%%%%%%%%%%%%%%%%%%%%%%%%%%%%%%%%%%%%%%

%%%%%%%%%%%%%%%%%%%%%%%%%%%%%%%%%%%%%%%%%%%%%%%%%%%%%%%%%%%%%%%%%%%%%%%%
%% NOTE: If you find that it says                                     %%
%%                                                                    %%
%%                           1 of ??                                  %%
%%                                                                    %%
%% at the bottom of your first page, this means that the AUX file     %%
%% was not available when you ran LaTeX on this source. Simply RERUN  %%
%% LaTeX to get the ``??'' replaced with the number of the last page  %%
%% of the document. The AUX file will be generated on the first run   %%
%% of LaTeX and used on the second run to fill in all of the          %%
%% references.                                                        %%
%%%%%%%%%%%%%%%%%%%%%%%%%%%%%%%%%%%%%%%%%%%%%%%%%%%%%%%%%%%%%%%%%%%%%%%%

%%%%%%%%%%%%%%%%%%%%%%%%%%%% Document Setup %%%%%%%%%%%%%%%%%%%%%%%%%%%%

% Don't like 10pt? Try 11pt or 12pt
\documentclass[10pt]{article}

% This is a helpful package that puts math inside length specifications
\usepackage{calc}
\usepackage{amsmath}%
\usepackage{amsfonts}%
\usepackage{amssymb}%
\usepackage{graphicx}
\usepackage{pictexwd,dcpic}
\usepackage{mathrsfs}
\usepackage{booktabs}
\usepackage{marvosym}
\usepackage{hyperref}


% Simpler bibsection for CV sections
% (thanks to natbib for inspiration)
\makeatletter
\newlength{\bibhang}
\setlength{\bibhang}{1em}
\newlength{\bibsep}
 {\@listi \global\bibsep\itemsep \global\advance\bibsep by\parsep}
\newenvironment{bibsection}%
        {\vspace{-\baselineskip}\begin{list}{}{%
       \setlength{\leftmargin}{\bibhang}%
       \setlength{\itemindent}{-\leftmargin}%
       \setlength{\itemsep}{\bibsep}%
       \setlength{\parsep}{\z@}%
        \setlength{\partopsep}{0pt}%
        \setlength{\topsep}{0pt}}}
        {\end{list}\vspace{-.6\baselineskip}}
\makeatother


% Layout: Puts the section titles on left side of page
\reversemarginpar

%
%         PAPER SIZE, PAGE NUMBER, AND DOCUMENT LAYOUT NOTES:
%
% The next \usepackage line changes the layout for CV style section
% headings as marginal notes. It also sets up the paper size as either
% letter or A4. By default, letter was used. If A4 paper is desired,
% comment out the letterpaper lines and uncomment the a4paper lines.
%
% As you can see, the margin widths and section title widths can be
% easily adjusted.
%
% ALSO: Notice that the includefoot option can be commented OUT in order
% to put the PAGE NUMBER *IN* the bottom margin. This will make the
% effective text area larger.
%
% IF YOU WISH TO REMOVE THE ``of LASTPAGE'' next to each page number,
% see the note about the +LP and -LP lines below. Comment out the +LP
% and uncomment the -LP.
%
% IF YOU WISH TO REMOVE PAGE NUMBERS, be sure that the includefoot line
% is uncommented and ALSO uncomment the \pagestyle{empty} a few lines
% below.
%

%% Use these lines for letter-sized paper
\usepackage[paper=letterpaper,
            %includefoot, % Uncomment to put page number above margin
            marginparwidth=1.2in,     % Length of section titles
            marginparsep=.05in,       % Space between titles and text
            margin=1in,               % 1 inch margins
            includemp]{geometry}

%% Use these lines for A4-sized paper
%\usepackage[paper=a4paper,
%            %includefoot, % Uncomment to put page number above margin
%            marginparwidth=30.5mm,    % Length of section titles
%            marginparsep=1.5mm,       % Space between titles and text
%            margin=25mm,              % 25mm margins
%            includemp]{geometry}

%% More layout: Get rid of indenting throughout entire document
\setlength{\parindent}{0in}

%% This gives us fun enumeration environments. compactitem will be nice.
\usepackage{paralist}

%% Reference the last page in the page number
%
% NOTE: comment the +LP line and uncomment the -LP line to have page
%       numbers without the ``of ##'' last page reference)
%
% NOTE: uncomment the \pagestyle{empty} line to get rid of all page
%       numbers (make sure includefoot is commented out above)
%
\usepackage{fancyhdr,lastpage}
\pagestyle{fancy}
%\pagestyle{empty}      % Uncomment this to get rid of page numbers
\fancyhf{}\renewcommand{\headrulewidth}{0pt}
\fancyfootoffset{\marginparsep+\marginparwidth}
\newlength{\footpageshift}
\setlength{\footpageshift}
          {0.5\textwidth+0.5\marginparsep+0.5\marginparwidth-2in}
\lfoot{\hspace{\footpageshift}%
       \parbox{4in}{\, \hfill %
                    \arabic{page} of \protect\pageref*{LastPage} % +LP
%                    \arabic{page}                               % -LP
                    \hfill \,}}

% Finally, give us PDF bookmarks
\usepackage{color,hyperref}
\definecolor{darkblue}{rgb}{0.0,0.0,0.3}
\hypersetup{colorlinks,breaklinks,
            linkcolor=darkblue,urlcolor=darkblue,
            anchorcolor=darkblue,citecolor=darkblue}

%%%%%%%%%%%%%%%%%%%%%%%% End Document Setup %%%%%%%%%%%%%%%%%%%%%%%%%%%%


%%%%%%%%%%%%%%%%%%%%%%%%%%% Helper Commands %%%%%%%%%%%%%%%%%%%%%%%%%%%%

% The title (name) with a horizontal rule under it
%
% Usage: \makeheading{name}
%
% Place at top of document. It should be the first thing.
\newcommand{\makeheading}[1]%
        {\hspace*{-\marginparsep minus \marginparwidth}%
         \begin{minipage}[t]{\textwidth+\marginparwidth+\marginparsep}%
                {\large \bfseries #1}\\[-0.15\baselineskip]%
                 \rule{\columnwidth}{1pt}%
         \end{minipage}}

% The section headings
%
% Usage: \section{section name}
%
% Follow this section IMMEDIATELY with the first line of the section
% text. Do not put whitespace in between. That is, do this:
%
%       \section{My Information}
%       Here is my information.
%
% and NOT this:
%
%       \section{My Information}
%
%       Here is my information.
%
% Otherwise the top of the section header will not line up with the top
% of the section. Of course, using a single comment character (%) on
% empty lines allows for the function of the first example with the
% readability of the second example.
\renewcommand{\section}[2]%
        {\pagebreak[2]\vspace{1.3\baselineskip}%
         \phantomsection\addcontentsline{toc}{section}{#1}%
         \hspace{0in}%
         \marginpar{
         \raggedright \scshape #1}#2}

% An itemize-style list with lots of space between items
\newenvironment{outerlist}[1][\enskip\textbullet]%
        {\begin{itemize}[#1]}{\end{itemize}%
         \vspace{-.6\baselineskip}}

% An environment IDENTICAL to outerlist that has better pre-list spacing
% when used as the first thing in a \section
\newenvironment{lonelist}[1][\enskip\textbullet]%
        {\vspace{-\baselineskip}\begin{list}{#1}{%
        \setlength{\partopsep}{0pt}%
        \setlength{\topsep}{0pt}}}
        {\end{list}\vspace{-.6\baselineskip}}

% An itemize-style list with little space between items
\newenvironment{innerlist}[1][\enskip\textbullet]%
        {\begin{compactitem}[#1]}{\end{compactitem}}

% An environment IDENTICAL to innerlist that has better pre-list spacing
% when used as the first thing in a \section
\newenvironment{loneinnerlist}[1][\enskip\textbullet]%
        {\vspace{-\baselineskip}\begin{compactitem}[#1]}
        {\end{compactitem}\vspace{-.6\baselineskip}}

% To add some paragraph space between lines.
% This also tells LaTeX to preferably break a page on one of these gaps
% if there is a needed pagebreak nearby.
\newcommand{\blankline}{\quad\pagebreak[2]}

% Uses hyperref to link DOI
\newcommand\doilink[1]{\href{http://dx.doi.org/#1}{#1}}
\newcommand\doi[1]{doi:\doilink{#1}}


%%%%%%%%%%%%%%%%%%%%%%%% End Helper Commands %%%%%%%%%%%%%%%%%%%%%%%%%%%

%%%%%%%%%%%%%%%%%%%%%%%%% Begin CV Document %%%%%%%%%%%%%%%%%%%%%%%%%%%%

\begin{document}

\makeheading{W\"oden B. Kusner}
\section{{\bf January 2019}} 

\section{Contact Information}
%
% NOTE: Mind where the & separators and \\ breaks are in the following
%       table.
%
% ALSO: \rcollength is the width of the right column of the table
%       (adjust it to your liking; default is 1.85in).
%
\newlength{\rcollength}\setlength{\rcollength}{2.5in}%
%
\begin{tabular}[t]{@{}p{\textwidth-\rcollength}p{\rcollength}}
\href{https://as.vanderbilt.edu/math/}{Department of Mathematics} & \Telefon\  +1 615 32(2) 6651
 \\
\href{https://www.vanderbilt.edu/}{Vanderbilt University}
                           & \Telefon\  +1 413 225 1323\\

1511 Stevenson Center  &\Letter\ \href{mailto:wkusner@gmail.com}{wkusner@gmail.com} \\    
Nashville, TN, 37240, USA      & \Mundus\ \href{http://wkusner.github.io/}{wkusner.github.io}\\
\end{tabular}
%\section{Objective}
%\section{Security Clearance}

\section{Citizenship}
USA

\section{Research Interests}
Discrete and Metric Geometry, Optimization, Geometry and Topology of Configuration Spaces, Analysis and Geometric Measure Theory, Integral Geometry, Combinatorics, Representation Theory.

\section{Education}
%
\href{http://www.pitt.edu/}{\textbf{University of Pittsburgh}},
Pittsburgh, PA USA
\begin{outerlist}
\item[] Ph.D., 
       %\href{http://www.mathematics.pitt.edu/}
            {Mathematics},
             August 2014 
                     \begin{innerlist}[]
    % \item Thesis: \emph{Problems in Discrete Geometry}
   %     \item Thesis Proposal: \emph{Problems in Discrete Geomerty}
    %    \item Candidacy Exam: \emph{}
        \item Advisor:
              \href{http://www.mathematics.pitt.edu/person/thomas-hales}
                   {Professor Thomas C. Hales}      
    %    \item Area of Study: Discrete and Metric Geometry
        \item Dissertation: \emph{Bounds on packing density via slicing}
        \end{innerlist}
\item[] M.A.,
             {Mathematics}, August 2010 \newline
\end{outerlist}

\href{http://www.haverford.edu/}{\textbf{Haverford College}},
Haverford, PA USA
\begin{outerlist}
\item[] B.S.,
      % \href{http://www.haverford.edu/}
             {Mathematics}, May 2007
                \begin{innerlist}[]
        \item Advisor: Professor John J. Flynn
        \item Thesis: \emph{Results in sphere packing density}
        \end{innerlist}
\end{outerlist}

\section{Academic Appointments}
\textbf{Visiting Assistant Professor} \hfill {8/2017 - Present}
\begin{innerlist}
\item[] \href{https://as.vanderbilt.edu/math/}{Department of Mathematics}\\
        \href{https://www.vanderbilt.edu/}{Vanderbilt University}
\end{innerlist}

\hfill\newline
\textbf{Postdoctoral Associate} \hfill {8/2017 - Present}
\begin{innerlist}
\item[] \href{https://math.vanderbilt.edu/dept/cca/index.html}{Center for Constructive Approximation}\\
        \href{https://www.vanderbilt.edu/}{Vanderbilt University}
\end{innerlist}

\hfill\newline
\textbf{FWF Postdoctoral Researcher} \hfill {9/2014 - 8/2017}
\begin{innerlist}

\item[] \href{https://finanz.math.tugraz.at/}{Institute of Analysis and Number Theory}\\
        \href{https://www.tugraz.at}{Graz University of Technology}
%\begin{innerlist}
%\item \href{http://www.nfs.gov/}{National Science Foundation} Cyber-Physical Systems (ENG, \href{http://www.nsf.gov/div/index.jsp?div=eccs}{ECCS})
%\begin{innerlist}
%\item Autonomous Driving in Mixed-Traffic Urban Environments (\href{http://www.nsf.gov/awardsearch/showAward.do?AwardNumber=0931669}{\#0931669})
%\item Automatic verification of hybrid systems
%\end{innerlist}
%\end{innerlist}
\end{innerlist}
\hfill\newline
\textbf{ Visiting Scholar} \hfill {Fall 2014}
 \begin{innerlist}
\item[] \href{https://www.esi.ac.at/}{Erwin Schr\"odinger International Institute}\\
University of Vienna 
\end{innerlist}

\hfill\newline
\textbf{ Visiting Scholar} \hfill {Spring 2015, Spring 2018}
 \begin{innerlist}
\item[] \href{https://icerm.brown.edu/home/index.php}{ICERM}\\
Brown University
\end{innerlist}
%\item Visitor, Brown University -- ICERM: Phase Transitions and Emergent Properties. February-May 2015.

\section{Papers} \begin{innerlist}[-] %{bibsection}


%\item W\"oden Kusner. Space frames and word games.  \emph{(in preparation)}
%\item W\"oden Kusner, Alden Walker. An algorithm for spherical discrepancy.  \emph{(in preparation)}
%\item Rob Kusner and W\"oden Kusner. Balanced Morse graphs.  \emph{(in preparation)}
%\item W\"oden Kusner. Quasi-onedimensional granular materials. \emph{(in preparation)}
%\item Greg Buck, Robert Kusner and W\"oden Kusner.  Combing curves. (\emph{in progress}), 2018.
%
\item with G. Buck and R. Kusner. Stopper knots. (\emph{in preparation}).
%
\item with G. Buck and R. Kusner. A Length-trading Gordian pair. (\emph{in preparation}).
%
\item with G. Dietler, E. Rawdon, R. Kusner and P. Szymczak. Chirality for crooked curves. (\emph{in preparation}).
%
\item with T. Hales. Packings of regular pentagons in the plane. To appear in \emph{Contemporary Mathematics (Festschrift for W. Kuperberg).}\\
\href{https://arxiv.org/abs/1602.07220}{https://arxiv.org/abs/1602.07220}  
%
\item with J. Brauchart, P. Grabner and J. Ziefle. Hyperuniform point sets on the sphere: probabilistic aspects, 2018.\\
\href{https://arxiv.org/abs/1809.02645}{https://arxiv.org/abs/1809.02645},  
%
\item with R. Kusner, J. Lagarias and S. Shlosman. Configuration spaces of equal spheres touching a given sphere: The twelve spheres problem. \emph{Bolyai Society Mathematical Studies: New Trends in Intuitive Geometry}, 2018.\\
\href{https://arxiv.org/abs/1611.10297}{https://arxiv.org/abs/1611.10297}  
%
\item with J. Brauchart and P. Grabner. Hyperuniform point sets on the sphere: deterministic aspects. \emph{Constr Approx}, %{\bfseries X}(X): 1-17,  
2018.\\
\href{https://arxiv.org/abs/1709.02613}{https://arxiv.org/abs/1709.02613}
%
\item with Y. Kallus. The local optimality of the double lattice packing. \emph{Discrete Comput Geom}, %{\bfseries56}(2): 449-471, 
2016.\\
\href{https://arxiv.org/abs/1509.02241}{https://arxiv.org/abs/1509.02241} 
%
\item On the densest packing of polycylinders in any dimension. \emph{Discrete Comput Geom}, %{\bfseries55}(3): 638-641, 
2016.\\
\href{https://arxiv.org/abs/1405.0497}{https://arxiv.org/abs/1405.0497} 
 %
\item An upper bound on packing density for circular cylinders of high aspect ratio. \emph{Discrete Comput Geom}, % {\bfseries52}(4): 964-972, 
2014.\\
\href{https://arxiv.org/abs/1309.6996}{https://arxiv.org/abs/1309.6996}

\end{innerlist} %{bibsection}

\section{Talks and Conferences} 
\begin{innerlist}[-]
\item  Topology and its Applications, WKU: \emph{Gordian configurations} (II). 7/17/18
\item  ICERM Seminar, Brown: \emph{Gordian configurations} (I). 4/11/18
\item ICERM Seminar, Brown: \emph{Computing discrepancy.} 3/9/18
\item Aspen Center for Physics, 6/2017
\item Montanuniversit\"at Leoben: \emph{Critical packings and the radius function.} 6/2/17
\item CEIM, Universidad de Cantabria: Optimal Point Configurations and Orthogonal Polynomials: \emph{Critical packings (in the sphere).} 4/22/17
\item JMM Special Session: Discrete Geometry and Convexity: \emph{Critical packings, rigidity, and the radius function.} 1/6/17
\item TU Graz, Fall School: Discrete Geometry and Topology: \emph{Critical packings, rigidity, and the radius function.} 9/30/16
\item  AIM Workshop on Soft Packings, Nested Clusters and Condensed Matter: \emph{Configurations of spheres.} 9/22/16
\item ICERM Workshop, Brown 9/16
\item ACG Seminar, Pittsburgh: \emph{Configurations of spheres.} 8/25/16
\item MCQMC, Stanford: \emph{Configurations of points with respect to discrepancy and uniform distribution.} 8/17/16
\item SRP, MSRI: \emph{Critical packings, rigidity, and the radius function.}  8/4/16
\item Institut Henri Poincar\'e Workshop, 6/16
\item Special Session on New Developments in Discrete and Intuitive Geometry, AMS Spring SE Sectional: \emph{Configurations of points with respect to discrepancy and uniform distribution.} 3/6/16
\item Advanced Topics Seminar, TU Graz: \emph{Configurations of spheres.} 1/22/16
\item Zahlentheoretisches Kolloquium, TU Graz: \emph{Problems with packing periodicity.} 12/11/15
\item  ICERM Seminar, Brown: \emph{Can rods pack space more densely than disks pack in the plane?} 4/28/15
\item ICERM Seminar, Brown University: \emph{Spherical discrepancy.}  4/9/15
%\item Visitor, Brown University -- ICERM: Phase Transitions and Emergent Properties. February-May 2015.
\item TU Graz: \emph{Computing spherical cap discrepancy: proof of concept.} 1/22/15
\item Guest Lecture, TU Graz: \emph{Introduction to packing problems.} 1/19/15
\item Large Structures Seminar, Aalto: \emph{Packing density bounds in higher dimensions.} 11/22/14
\item ESI Workshop: \emph{A brief analysis of regular pentagon packings in the plane.} 8/27/14
%\item Visitor, University of Vienna -- ESI: Minimal Energy Point Sets.  October 2014.
\item Researcher, IAS -- PCMI: Mathematics and Materials. 6/14
\item Oberwolfach: \emph{Packing polycylinders.} 6/14
\item Dissertation Defense, Pittsburgh: \emph{Bounds on packing density via slicing.} 5/22/14
\item Seminar, TU Graz: \emph{Packing density bounds via slicing.} 5/8/14
\item Erd\H{o}s Memorial Lectures, Memphis: \emph{%Revisiting Bezdek and Kuperberg: A sharp upper bound for the packing
Polycylinder density in higher dimensions.} 3/14
\item Fields Institute Workshop in Discrete Geometry, Fields Institute. 11/13
\item GSS, Pittsburgh: \emph{Some packing problems and an upper bound.}  3/28/13
\item A\&SGraduate Expo, Pittsburgh:  \emph{Packing cylinders with high aspect ratio.}  3/23/13
\item Algebra, Combinatorics and Geometry Seminar, Pittsburgh: \emph{An upper bound on packing density for circular cylinders with high aspect ratio.} 2/12/13
\item Topological Dynamics Workshop, Newton Institute:  \emph{Packing circular cylinders.}  11/12
\item IMA Summer School in Topological Methods, Penn. `11
\item Graduate Algebra, Combinatorics and Geometry Seminar, Pittsburgh:
 \begin{innerlist}
		\item \emph{The Jones Polynomial and the Kauffman Bracket }
		\item \emph{Category Theory V (Representable Functors)}
		\item \emph{Category Theory IV (Limits Informally/Formally)}
		\item \emph{Category Theory III (Slice and Comma Categories)}
		\item \emph{Category Theory II (Products and Limits)}
\end{innerlist}
\item Senior Thesis Defense, Haverford: \emph{Results in sphere packing density.} 5/07
\end{innerlist}

%\section{Books in Preparation} \begin{bibsection}
%    \item Pavlic, T.P., B.W.~Andrews, K.M.~Passino, and T.A.~Waite.
%        \emph{Foraging Theory for Engineering}.
%\end{bibsection}

%\section{Referee for Journals} \begin{loneinnerlist}
 %  \item \emph{IEEE Transactions on Signal Processing}
   
%\end{loneinnerlist}

\section{Honors \& Awards}
%\href{http://www.nsf.gov/}{National Science Foundation}
%\begin{bibsection}
\begin{innerlist}[-]
\item Featured in \href{http://wkusner.github.io/assets/DiePresse2017422.pdf}{Die Presse: Science and Innovation}, `17
\item University of Pittsburgh Honors Convocation `13, `14
\item Outstanding Presentation: University Graduate Expo `13
\item University of Pittsburgh Irvis Fellowship `09, `12
\item Bronze Presidential Service Award for AmeriCorps Volunteer Service `08
\end{innerlist}
%\end{bibsection}

\section{Teaching Experience}
\href{https://www.vanderbilt.edu/}{\textbf{Vanderbilt University}},
Nashville, TN USA

\begin{outerlist}
\item[] \textit{Visiting Assistant Professor}% \rm{(NTT)}}%
\hspace{\footpageshift}\textrm{8/17 - Present}
\vspace{-.5em}
\item[] \textit{Certificate In College Teaching: Parts I \& II}
    \begin{innerlist}[-]
       \item Instructor for MATH 3641: Statistical Inference \hfill Spring `19
              \item Supervisor for MATH 5641: Graduate Statistics \hfill Spring `19
     \item Instructor for MATH 2300: Vector Calculus \hfill Fall `18
     \item Instructor for MATH 1010:  Prob. \& Stat. Inference I \hfill Fall `18
          \item Mentor/Supervisor for MATH 1010 Undergraduate TAs \hfill Fall `18
      \item Instructor for MATH 3641:  Statistical Inference \hfill Spring `18
                    \item Supervisor for MATH 5641: Graduate Statistics \hfill Spring `18
      \item Instructor for MATH 1011:  Prob. \& Stat. Inference II \hfill Spring `18
     \item Instructor for MATH 1010:  Prob. \& Stat. Inference I \hfill Fall `17
     \item Mentor/Supervisor for MATH 1010 Undergraduate TAs\hfill Fall `17
       \end{innerlist}~
\end{outerlist}



%
\href{http://www.tugraz.at}{\textbf{Graz University of Technology}},
Graz, AT
%
\begin{outerlist}
\item[] \textit{Lehrbeauftragter}%
\hspace{\footpageshift}\textrm{10/14 - 1/15, 3/16 - 6/16} %March-June 2016
    \begin{innerlist}[-]
     \item Instructor for MAT.670:  Packings, Lattices and Configurations\hfill Summer `16
     \item Assistant for MAT.902: H\"ohere Analysis   \hfill Winter `14 
    \end{innerlist}~
\end{outerlist}

%
\newpage
\href{http://www.pitt.edu}{\textbf{University of Pittsburgh}},
Pittsburgh, PA USA
\begin{outerlist}

\item[] \textit{Teaching Fellow}%
\hspace{\footpageshift}\textrm{9/10 - 12/11}
    \begin{innerlist}[-]
     \item Assistant for Math 0220: Calculus I  (3 sections) \hfill Fall `11
     \item Assistant for Math 2700: Graduate Topology \hfill Fall `11
    \item Assistant for Math 1700: Topology \hfill Spring `11
    \item Assistant for Math 1410: Foundations of Mathematics \hfill Spring `11
    \item Assistant for Math 1250: Abstract Algebra 2 \hfill Spring `11
        \item Assistant for Math 0230: Calculus 2 \hfill Fall `10

         %   \item Responsible for 0.5~hour lecture and supervision of
         %       3.5~hour laboratory where senior undergraduate students
         %       combine
         %       \href{http://www.mathworks.com/products/simulink/}{Simulink},
         %       with \href{http://www.dspaceinc.com/}{dSPACE} RTI1104
         %       real-time control hardware and software to do analysis
         %       and control implementation for linear systems

         %   \item Developed supplementary course
         %       material, including a course web page archived at
         %       \href
         %       {http://www.tedpavlic.com/teaching/osu/ece557}
         %       {\texttt{http://www.tedpavlic.com/teaching/osu/ece557}}

        \item Assistant for Math 0220:  Calculus 1  (2 sections) \hfill Fall `10

         %   \item Responsible for 0.5~hour lecture and supervision of
         %       3.5~hour laboratory where graduate students and senior
         %       undergraduate students combine
         %       \href{http://www.mathworks.com/products/simulink/}{Simulink},
         %       with \href{http://www.dspaceinc.com/}{dSPACE} RTI1104
         %       real-time control hardware and software to do analysis
         %       and advanced control implementation for linear and
         %       non-linear systems

         %   \item Developed supplementary course
         %       material, including a course web page archived at
         %       \href
         %       {http://www.tedpavlic.com/teaching/osu/ece758}
         %       {\texttt{http://www.tedpavlic.com/teaching/osu/ece758}}
      
    \end{innerlist}

\item[] \textit{Teaching Assistant}%
    \hspace{\footpageshift} \textrm{9/09 - 8/10}
    \begin{innerlist}[-]
        \item Instructor for Math 0120: Business Calculus \hfill Summer `10

        \item Assistant for Math 0120:  Business Calculus (3 sections)  \hfill Spring `10 

               \item Assistant for Math 0240:  Calculus 3 (3 sections) \hfill Fall `09 

            
          %  \item Responsible for 0.5~hour lecture and supervision of
              %  3.5~hour laboratory where sophomore undergraduate
                %students learn learn how to use basic laboratory
                %equipment to study properties of electronic circuits

            %\item Developed supplementary course
            %    material, including a course web page archived at
            %    \href
            %    {http://www.tedpavlic.com/teaching/osu/ece209}
            %    {\texttt{http://www.tedpavlic.com/teaching/osu/ece209}}
       
        \end{innerlist}
        
 Sample material and student evaluations available upon
    request

%\item[] \textit{Graduate Student}%
%        \hfill \textbf{June 2004 to present}
%\begin{innerlist}
%\item \href{http://www.gradsch.osu.edu/Content.aspx?Content=44&itemid=2}
%           {Dean's Distinguished University Fellow}
%      (June 2004 to present)
%        \begin{innerlist}
%        \item[] Includes M.S.~and Ph.D.~research and course work.
%        \end{innerlist}
%\item \href{http://www.nsfgk12.org/}
%           {National Science Foundation GK-12 Fellow}
%      (September 2006 to October 2007)
%        \begin{innerlist}
%        \item[] Developed, implemented, and evaluated daily fourth grade
%                science lessons for a local inner-city public school
%                class.
%        \end{innerlist}
%\end{innerlist}

%{\item[] \textit{Instructor}%
%        \hfill \textbf{March 2002 to June 2004}
%\begin{innerlist}
%\item Member of \href{http://feh.eng.ohio-state.edu/}
%                     {Fundamentals of Engineering for Honors}
%      instructional team.
%\item Special graduate teaching appointment as undergraduate.
%\item Lectured weekly laboratory on engineering fundamentals (ENG~H191,
%        H192, and~H193).
%\item Trained in-class undergraduate teaching assistants in laboratory
%        procedure.
%\item Graded weekly lab reports and provided laboratory exams.
%\end{innerlist}

%\item[] \textit{Teaching Assistant}%
%        \hfill \textbf{September 2000 to March 2002}
%\begin{innerlist}
%\item Assisted \href{http://feh.eng.ohio-state.edu/}
%                    {Fundamentals of Engineering for Honors}
%      instructional team.
%\item Provided in-class support to first-year engineering students (ENG
%        H191, H192, and H193).
%\item Graded daily assignments on programming and drafting.
%\item Developed on-line journal submission and report system for Physics
%    Education Research Group (PERG).
%\end{innerlist}

%\item[] \textit{Undergraduate Researcher}%
%        \hfill \textbf{September 2000 to March 2002}
%\begin{innerlist}
%\item Participated in the
%        \href{http://www.cse.ohio-state.edu/europa/}{Europa
%        Undergraduate Research Forum}, a part of the
%        \href{http://www.cse.ohio-state.edu/rsrg/}{Reusable Software
%        Research Group}.
%\item Worked to improve undergraduate education of component based
%        software engineering topics.
%\item Researched needed changes to RESOLVE/C++ implementation for
%        ANSI/C++ compliance.
%\end{innerlist}

%\item[] \textit{Grader}%
%        \hfill \textbf{September 2001 to December 2001}
%\begin{innerlist}
%\item Graded daily electromagnetics assignments (ECE~311).
%\end{innerlist}

%\item[] \textit{Undergraduate Student}%
%        \hfill \textbf{September 1999 to June 2004}
\end{outerlist}

\section{Professional Service}
%\href{http://www.pitt.edu}{\textbf{University of Pittsburgh}},
%Pittsburgh, PA USA
%\begin{bibsection}
%\begin{outerlist}[]
\begin{innerlist}[-]
\item Active referee and reviewer for various journals and scientific bodies.
\item Dissertation Committee for Oleksandr Vlasiuk (Vanderbilt)
\item Research Mentor for Jonas Zifle (Graz)
\item Organizer: Shanks Workshop (Vanderbilt University) `19 (upcoming)
\item Organizer: Computational Analysis Seminar (Vanderbilt University) `18 - `19
\item Organizer: From the Fundamental Lemma to Dis. Geo. to Formal Verification `18
\item Representative: Dietrich School of Arts and Sciences Council, `12 - `14
\item Delegate: Arts and Sciences Graduate Student Organization, `11 - `14
\item President: Mathematics Graduate Student Organization, `13 - `14
\item Treasurer: Mathematics Graduate Student Organization, `11 - `13
\item Treasurer: SIAM University of Pittsburgh Chapter, `10 - `11
\item Organizer: Graduate Seminar in Algebra, Combinatorics and Geometry, `10 - `11
\end{innerlist}
%\end{outerlist}
%\end{bibsection}


%\item[] \textit{Hardware R\&D Intern for Multifunction DAQ}%
%        \hfill \textbf{June 2003 to September 2003}
%\begin{innerlist}
%\item Designed final verification testing fixture for use with STC2 MIO
%        products.
%\item Designed and executed study of the effect of varying burn-in time
%        on long-term drift of common industry voltage references.
%\end{innerlist}

%\item[] \textit{Hardware R\&D Intern for Multifunction DAQ}%
%        \hfill \textbf{June 2002 to September 2002}
%\begin{innerlist}
%\item Designed and performed validation tests on new 16-bit 800 kHz
%        NI-6120 SMIO DAQ board.

%\item Designed high quality filter/amplifier source for use with NI-5411
%        arbitrary function generator.
%\end{innerlist}

%\end{outerlist}

%\blankline

%\textbf{\href{http://www.ibm.com/}{IBM} Network Storage},
%Research Triangle Park, North Carolina USA
%\begin{outerlist}

%\item[] \textit{Core Systems Software Developer for FlexNAS}%
%        \hfill \textbf{June 2001 to September 2001}
%\begin{innerlist}
%\item Designed and implemented high-availability, redundant internode
%        communications subsystem.
%\item Participated in software development of various vital box
%        services.
%\end{innerlist}

%\end{bibsection}

%\blankline

%\href{http://www.calltech.com/}{\textbf{CallTech Communications}},
%Columbus, Ohio USA
%\begin{outerlist}

%\item[] \textit{Information Technology Systems Engineer}%
%        \hfill \textbf{June 1997 to May 2001}
%\begin{innerlist}
%\item Responsible for the acquisition, setup, maintenance, and
%        administration of all Internet hardware and software supporting
%        \href{http://www.netwalk.com/}{NetWalk} Internet service
%        and web presence provider.
%\item Designed and implemented state of the art open source
%        high-availability load balancing system supporting thousands of
%        virtual servers.
%\item Developed software call center support software for clients such
%        as CompuServe, AOL, and Priceline.
%\end{innerlist}

%\end{outerlist}

\blankline

{\section{Technical\hspace{2em} Skills}
%
%Extensive hardware and software experience in networking, %information
%        technology, and analog and digital electronics
%\blankline
%\href{http://www.mathworks.com/products/matlab/}{\textsc{Matlab}}
%        experience: linear algebra, Fourier transforms,
%        nonlinear numerical methods, polynomials, statistics,
%        $N$-dimensional filters, visualization
%\blankline
%\href{http://www.mathworks.com/products/matlab/}{\textsc{Matlab}}
%        toolboxes: communications, control system, filter
%        design, genetic algorithm and direct search, signal processing,
%        system identification
%\blankline
%Embedded Systems: Software and hardware development with several MCU and
%        DSP platforms (e.g., Motorola MCU's, Texas Instruments DSP's, Atmel
%        ATmega MCU's, Microchip PIC MCU's, and others)
%\blankline
%Instrumentation and Control:
%        \href{http://www.dspaceinc.com/}{dSPACE} hardware (e.g.,
%        RTI1104) and Control Desk software,
%        \href{http://www.mathworks.com/products/simulink/}{Simulink},
%        \href{http://www.ni.com/}{LabVIEW} and other
%        \href{http://www.ni.com}{National Instruments}
%        control and data acquisition hardware and software (e.g., MIO,
%        SMIO, DSA, DMM, and others)
%\blankline
%Analog and Digital Electronics: Bipolar and FET implementations of
%        continuous and switched amplifiers, modulators, and filters
%\blankline
%Programming: C, C$+$$+$, Java, JavaScript, Pascal, Perl, PHP, Lisp, UNIX
%        shell scripting, GNU make, AppleScript, SQL, DVCS (Mercurial,
%        git), VCS (RCS, CVS, SVN, SCCS), and others
%\blankline
%Information Technology: Networking (UDP, TCP, ARP, DNS, Dynamic
%        routing), Service (Apache, SQL, MediaWiki, POP, IMAP, SMTP,
%        application-specific daemon design)
%\blankline
 \TeX{} (\LaTeX{}, B\textsc{ib}\TeX{}), Mathematica.  German (AP, B2), Spanish (AP).

%        most common productivity packages (for Windows, OS X, and Linux
%        platforms), Vim

%\blankline

%Computer-Aided Design: Cadence OrCAD, NI Multisim, SPICE, pst-circ

%\blankline

%Operating Systems: Microsoft Windows family, Apple OS X, Linux, BSD,
%        IRIX, AIX, Solaris, and other UNIX variants
        
       

%\section{Mathematical Expertise}
%
%Real and Complex Analysis, Measure Theory, Differential Geometry, Game Theory, Graph Theory, Combinatorics, Discrete Geometry, Representation Theory, Group and Ring Theory, Langlands-Tunnell, Algebraic Geometry, Topology, Linear Algebra, 

%\section{Engineering Expertise}
%
%Control: Linear and Nonlinear Systems Theory, Feedback, %Variable
%        Structure Systems and Sliding Modes, Distributed and Intelligent
%        Control, Dynamic Optimization, Bio-mimicry

\blankline

%Communications and Signal Processing: Probability, Random Variables,
%Stochastic Processes, Estimation, Networks

%\section{Biological Expertise}
%
%Behavioral Ecology: Foraging Theory, Cooperation, Impulsiveness,
%Evolution

%\section{Application Areas}
%
%Autonomous and Unmanned Vehicles, Flexible Manufacturing Systems,
%Distributed Power Generation, Intelligent Lighting, Power Demand
%Response, Microgrids, Smart Grids
\section{References}\vspace{-.8\baselineskip}\\
%Please contact me for references.
\begin{tabular}{lr}
%% Referee 1
\begin{minipage}[t]{2.5in}
Prof.\ Thomas C. Hales\\
Andrew Mellon Professor\\
University of Pittsburgh\\
Pittsburgh PA 15260\\
%%\Telefon\ +00 1 234 5678\\
\Letter\ \href{mailto:hales@pitt.edu}{hales\textrm{@}pitt.edu}
\end{minipage}
%&
%% Referee 2
%\begin{minipage}[t]{2.5in}
%Prof.\ N. John Cooper\\
%Bettye J. and Ralph E. Bailey Dean\\
%University of Pittsburgh\\
%Pittsburgh PA, 15260\\
%%\Telefon\ +00 1 234 5678\\
%\Letter\ \href{mailto:cooper@pitt.edu}{cooper\textrm{@}pitt.edu}
%\end{minipage}
%\\
%% Referee 2
&
\begin{minipage}[t]{2.5in}
Univ.-Prof.\ Peter J. Grabner\\
Institute of Analysis and Number Theory\\
Graz University of Technology\\
8010 Graz Austria\\
%%\Telefon\ +00 1 234 5678\\
\Letter\ \href{mailto:peter.grabner@tugraz.at }{peter.grabner\textrm{@}tugraz.at }
\end{minipage}
\\
\\

% Referee 2

\\ % Additional newline for spacing.
% Referee 2
%\begin{minipage}[t]{2.5in}
%Prof.\ Karoly Bezdek\\
%Canada Research Chair\\
%University of Calgary\\
%Calgary AB Canada\\
%%\Telefon\ +1 (403) 220-6919\\
%\Letter\ \href{mailto:kbezdek@ucalgary.ca}{kbezdek{@}ucalgary.ca}
%\end{minipage}
%&
\begin{minipage}[t]{2.5in}
Prof.\ Douglas Hardin\\
Mathematics and Informatics\\
Vanderbilt University\\
Nashville TN 37240\\
%\Telefon\ +1 (403) 220-6919\\
\Letter\ \href{mailto:doug.hardin@vanderbilt.edu}{doug.hardin{@}vanderbilt.edu}
\end{minipage}
&
\begin{minipage}[t]{2.5in}
Henry Cohn \\
Principal Researcher\\
Microsoft Research New England\\
Cambridge MA 02142\\
%%\Telefon\ +00 1 234 5678\\
\Letter\ \href{mailto:cohn@microsoft.com}{cohn\textrm{@}microsoft.com}
\end{minipage}
%
%
%
%
%
%\\
%\\
%\\ % Additional newline for spacing.
%%% Referee 2
%
%%% Referee 2
%\begin{minipage}[t]{2.5in}
% Dr. Jeffrey Wheeler\\
%Lecturer\\
%University of Pittsburgh\\
%Pittsburgh PA 15260\\
%%\Telefon\ +00 1 234 5678\\
%\Letter\ \href{mailto:jwheeler@pitt.edu}{jwheeler\textrm{@}pitt.edu}
%\end{minipage}
%
%
%%&
%%% Referee 2
%%\begin{minipage}[t]{2.5in}
%%Prof.\ N. John Cooper\\
%%Bettye J. and Ralph E. Bailey Dean\\
%%University of Pittsburgh\\
%%Pittsburgh PA, 15260\\
%%%\Telefon\ +00 1 234 5678\\
%%\Letter\ \href{mailto:cooper@pitt.edu}{cooper\textrm{@}pitt.edu}
%%\end{minipage}
%
%%&
%%\begin{minipage}[t]{2.5in}
%%Prof.\ Jason DeBlois\\
%%Assistant Professor\\
%%University of Pittsburgh\\
%%Pittsburgh PA, 15260\\
%%%\Telefon\ +00 1 234 5678\\
%%\Letter\ \href{mailto:jdeblois@pitt.edu}{jdeblois\textrm{@}pitt.edu}
%%\end{minipage}
%
%%&
%%% Referee 2
%%\begin{minipage}[t]{2.5in}
%%Prof.\ Huiqiang Jiang\\
%%Associate Professor\\
%%University of Pittsburgh\\
%%Pittsburgh PA, 15260\\
%%%\Telefon\ +00 1 234 5678\\
%%\Letter\ \href{mailto:hqjiang@pitt.edu}{hqjiang\textrm{@}pitt.edu}
%%\end{minipage}
\end{tabular}
\end{document}

%%%%%%%%%%%%%%%%%%%%%%%%%% End CV Document %%%%%%%%%%%%%%%%%%%%%%%%%%%%%
